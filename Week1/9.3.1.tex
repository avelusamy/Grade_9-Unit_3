\documentclass{article}
\usepackage[utf8]{inputenc}
\usepackage{amsmath, amsfonts,fullpage,enumerate}

\title{Latex Homework 9th Grade\\ Unit 3 - Methods of Proof - Techniques \\ Week 1 - Induction}
\author{Dr. Chapman and Dr. Rupel}
\date{Edited \today}

\begin{document}

\maketitle

\section{}
    Explain what is wrong with the following proof:\\
    
    \noindent Theorem: Any collection of points in the plane lie along a line.\\
    
    \noindent Proof:
    Let $P(n)$ be the statement ``any collection of $n$ points in the plane lie along a line''.
    Every point sits on a line, so $P(1)$ is true.
    Assume $P(n)$ is true and consider a collection of $n+1$ points in the plane.
    The first $n$ of these points and the last $n$ of these points each lie along a line by assumption.
    But these share the middle $n-1$ points which lie on both lines and so the two lines must be the same.
    It follows that $P(n+1)$ is true.
    By the Principle of Mathematical Induction, $P(n)$ is true for all $n\ge 1$.

\section{}
    Let $p(x)=ax^5+bx^4+cx^3+dx^2+ex+f$ be a degree 5 polynomial.
    \begin{enumerate}[(a)]
        \item Write down a system of equations that states $\sum\limits_{k=1}^n k^4=p(n)$ for $n=0,1,2,3,4,5$.
        \item Solve the system of equations from (a) to find the polynomial $p(x)$ explicitly.
        \item Prove by induction that $\sum\limits_{k=1}^n k^4=p(n)$ for all $n\ge 0$.\\
        \emph{Hint: It might simplify your proof a little if you factor $p(n)$ first.}
    \end{enumerate}
    
\section{}
    Define the sequence $f_n$ for $n\ge1$ by the recursion $f_{n+1}=f_n+f_{n-1}$ with $f_1=f_2=1$.
    Prove by strong induction that $f_n=\frac{1}{\sqrt{5}}\left(\left(\frac{1+\sqrt{5}}{2}\right)^n-\left(\frac{1-\sqrt{5}}{2}\right)^n\right)$.\\
    \emph{Hint: $\frac{1+\sqrt{5}}{2}=1-\left(\frac{1-\sqrt{5}}{2}\right)$ and $\left(\frac{1+\sqrt{5}}{2}\right)\left(\frac{1-\sqrt{5}}{2}\right)=-1$}

\newpage

\section{}
    Consider the statement $P(n)$ given by ``$12|(n^4-n^2)$''.
    \begin{enumerate}[(a)]
        \item Write the beginning steps of an induction proof.
        What expression would you have to show is divisible by $12$ in order to complete the proof?
        Observe that this is not obvious and would require deeper considerations and possible casework.
        \item Show instead that $P(n)$ being true implies $P(n+6)$ is true.
        Use this to construct a proof that $P(n)$ is true for all $n\ge1$ by strong induction.
    \end{enumerate}
    
\end{document}
